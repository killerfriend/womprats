%%MaD.tex - Notes taken for Materials and Devices Lecture
%%Author: Andy Goetz
%%Date Modified: 10-7-09
%%License: Ask me before reproducing/modifying, etc.


\documentclass{article}

%Make sure you have the file ShumanNote.scy in the same directory as
%this one. It has contains the style sheet for ECE111, and is needed
%to standardize the layout of LateX documents created for the class.
\usepackage{ShumanNotes} 
\usepackage{tikz}
\usepackage{program}
\usepackage{listings}
\pdfpagewidth 8.5in 
\pdfpageheight 11in

%This package is used to line up pictures 
\usepackage{graphicx}
\usepackage{fancyvrb}
\usepackage{listings}
%allows cursive font
%\usepackage{amsmath}

%allows hyperlinks 
%\usepackage{hyperref}

\newcommand{\HRule}{\rule{\linewidth}{0.5mm}} 

\lhead{Homework 5}

\begin{document}

%% These commands allow me to use cursive letter for things such as
%% length.  Note that on ubuntu linux, this required installation of
%% the package 'texlive-fonts-extra'. 
%% Taken from
%% http://www.latex-community.org/forum/viewtopic.php?f=5&t=1404&start=0
\newenvironment{frcseries}{\fontfamily{frc}\selectfont}{}
\newcommand{\textfrc}[1]{{\frcseries#1}}
\newcommand{\mathfrc}[1]{\text{\textfrc{#1}}}

\section{Overview}
The audio synthesizer is able to generate an audio output based on the user input.  The functionality of the circuit can be broken down into several subcomponents.  These include a microcontroller to interface with the external components, a power supply, EEPROM to dynamically store/load pre-defined user settings, a LCD output, a DAC, and an audio amplifier to provide more output power to the speaker.

\section{T-16 Audio Synthesizer Level 0 Diagram}
\centerimage{\includegraphics[width=5in]{synth0.png}}{Level 0 Diagram}{fullzero}

\begin{tabular}{|p{1in}|p{5in}|}
\hline
\emph{Module} & T-16 Audio Synthesizer \\
\hline
\emph{Inputs}& 5VDC Power: Unregulated 5 VDC power from a wall-wart power supply.\\
	     & Channel Interface: Six analog/digital paired signals that modify parameters that control the synthesizer software. \\
      	     & UI Buttons: Up/Down/Left/Right/Ok/Aux buttons that allow you to navigate the Menu UI on the LCD and modify tunable parameters in the synthesizer software.\\
	     & Volume Control: Master output volume control of the internal speaker.\\
\hline
\emph{Outputs}& Audio: \\ 
	      & LCD: Shows the Menu user interface which presents the user with controllable parameters that affect the sound synthesis. It also presents the logo on startup. \\
\hline
\emph{Functionality}& Does stuff \\
\hline
\end{tabular}

\section{T-16 Audio Synthesizer Level 1 Diagram}
\centerimage{\includegraphics[width=5in]{synth1.png}}{Level 1 Diagram}{fullone}

\newpage

\section{LCD Level 0 Diagram}
\centerimage{\includegraphics[width=5in]{lcd.png}}{Level 0 Diagram}{LCD}

\begin{tabular}{|p{1in}|p{5in}|}
\hline
\emph{Module} & LCD \\
\hline
\emph{Inputs}& SPI: Receives image data from microcontroller.\\
	     & PWM: Backlight brightness control 0-100\% duty cycle.\\
	     & Power: 3.3 VDC Power.\\
\hline
\emph{Outputs}& Image: Black and white image on 102x64 raster LCD.\\
	      & Backlight: LED light to illuminate the LCD.\\ 
\hline
\emph{Functionality}& Display the user interface and state of the synthesizer.\\
\hline
\end{tabular}

\section{Power Supply Level 0 Diagram}
\centerimage{\includegraphics[width=5in]{pwrsply.png}}{Level 0 Diagram}{PSU}

\begin{tabular}{|p{1in}|p{5in}|}
\hline
\emph{Module} & Power Supply \\
\hline
\emph{Inputs}& 5VDC: Unregulated power from 5 VDC wall-wart power supply.\\
	     & Switch: On/Off Power Switch.\\
\hline
\emph{Outputs}& 3v3DC: Regulated 3.3 VDC Power.\\
	      & Status: LED to display if the input power is available.\\ 
\hline
\emph{Functionality}& The power supply provides regulated 3.3 VDC power to all the subsystems of the audio synthesizer.\\
\hline
\end{tabular}

\section{EEPROM Level 0 Diagram}
\centerimage{\includegraphics[width=5in]{eeprom.png}}{Level 0 Diagram}{eeprom}

\begin{tabular}{|p{1in}|p{5in}|}
\hline
\emph{Module} & EEPROM \\
\hline
\emph{Inputs}& Power: 3.3 VDC Power\\
	     & I2C In: Read data from microcontroller \\
\hline
\emph{Outputs}& I2C Out: Write data to microcontroller \\ 
\hline
\emph{Functionality}& Holds user settings for the synthesizer state, as well as the Womprats logo for startup\\
\hline
\end{tabular}

\section{Microcontroller Level 0 Diagram}
\centerimage{\includegraphics[width=5in]{microcont.png}}{Level 0 Diagram}{micro}

\begin{tabular}{|p{1in}|p{5in}|}
\hline
\emph{Module} & Microcontroller \\
\hline
\emph{Inputs}& Power: 3.3 VDC Power\\
	     & I2C: Retrieve data stored in EEPROM\\
	     & UI Buttons: Up/Down/Left/Right/OK/Aux button controls for UI menu interface and entering audio synthesizer settings.\\
	     & Channel Interface: Six analog and digital signal lines that supply parameter control to the audio synthesizer.\\
\hline
\emph{Outputs}& SPI$_0$: Send digital audio samples to DAC.\\
	      & SPI$_1$: Send image data to LCD.\\
	      & I2C: Store data in EEPROM.\\ 
\hline
\emph{Functionality}& Implements a software synthesizer that generates an audio signal, shows it's current settings via an LCD screen, is controllable through tunable parameters in the UI and from analog/digital signals on its channel inputs. The tunable parameters are stored in an EEPROM memory that is recalled on startup.\\
\hline
\end{tabular}

\section{DAC Level 0 Diagram}
\centerimage{\includegraphics[width=5in]{dac.png}}{Level 0 Diagram}{dac}

\begin{tabular}{|p{1in}|p{5in}|}
\hline
\emph{Module} & DAC \\
\hline
\emph{Inputs}& SPI$_0$: Audio samples received from the microcontroller\\
	     & Power: 3.3 VDC power\\
\hline
\emph{Outputs}& Audio: 0-3.3V audio signal \\ 
\hline
\emph{Functionality}& The DAC takes the digital audio sample from the microcontroller and converts an analog 0-3.3V audio signal.\\
\hline
\end{tabular}

\section{Audio Amplifier Level 0 Diagram}
\centerimage{\includegraphics[width=5in]{audioamp.png}}{Level 0 Diagram}{audio}

\begin{tabular}{|p{1in}|p{5in}|}
\hline
\emph{Module} & Audio Amplifier \\
\hline
\emph{Inputs} & Power: 3.3 VDC Power\\
	      & Audio: 0-3.3V audio signal from DAC\\
	      & Volume: Controls the audio output volume. Logarithmic control.\\
\hline
\emph{Outputs}& Line out: Unamplified audio output to external line-out.\\ 
	      & Speaker: Amplified audio output to internal speaker.\\
\hline
\emph{Functionality}& Provide an amplified audio out that can be heard over the internal speaker or line-out to an external audio device.\\
\hline
\end{tabular}

\section{EEPROM UML Use Cases}

\subsection{Save Synth State}
\begin{tabular}{|p{1in}|p{5in}|}
\hline
\textbf{Use Case} & Save\\
\hline
\textbf{Description} & Saves the current synthesizer state into the synthesizer core memory.\\
\hline
\textbf{Actors} & I$^2$C, EEPROM, synthesizer core\\
\hline
\textbf{Assumptions} & There will be the correct amount of power supplied to the EEPROM and microcontroller.  In addition, there must be usable storage space to save in EEPROM.\\
\hline
\textbf{Steps} & \begin{enumerate}
\item Receive synth state data pointer from synth core.
\item Iterate over every byte of state data and send to the EEPROM via I$^2$C.
\end{enumerate}\\
\hline
\textbf{Issues} & Synth state has yet to be defined.\\
& Funtionality has yet to be implemented.\\
& EEPROM memory map has yet to be defined.\\
\hline
\end{tabular}

\subsection{Load Synth State}
\begin{tabular}{|p{1in}|p{5in}|}
\hline
\textbf{Use Case} & Load\\
\hline
\textbf{Description} & Loads pre-saved synthesizer state into synthesizer core memory.\\
\hline
\textbf{Actors} & I$^{2}$C, EEPROM, synthesizer core\\
\hline
\textbf{Assumptions} & There will be the correct amount of power supplied to the EEPROM and microcontroller.  In addition, there will be a valid state to load.\\
\hline
\textbf{Steps} & \begin{enumerate}
\item Read synth state data from I$^2$C EEPROM.
\item Pass synth state data pointer to synth core.
\end{enumerate}\\
\hline
\textbf{Issues} & Synth state has yet to be defined.\\
& Funtionality has yet to be implemented.\\
& EEPROM memory map has yet to be defined.\\
\hline
\end{tabular}

\subsection{Load Splash Screen Image}
\begin{tabular}{|p{1in}|p{5in}|}
\hline
\textbf{Use Case} & Splash Screen Image\\
\hline
\textbf{Description} & Loads the Womprats Logo splash screen into memory.\\
\hline
\textbf{Actors} & LCD driver, EEPROM, I$^2$C\\
\hline
\textbf{Assumptions} & There will be the correct amount of power supplied to the EEPROM and microcontroller.  In addition, the image bits to be loaded will always remain intact.\\
\hline
\textbf{Steps} & \begin{enumerate}
\item Read logo data from I$^2$C EEPROM.
\item Pass logo data to LCD driver
\end{enumerate}\\
\hline
\textbf{Issues} & How the logo data is loaded into EEPROM has yet to be determined.\\
& Funtionality has yet to be implemented.\\
& EEPROM memory map has yet to be defined.\\
\hline
\end{tabular}

\end{document}
