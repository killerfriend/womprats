%%MaD.tex - Notes taken for Materials and Devices Lecture
%%Author: Andy Goetz
%%Date Modified: 10-7-09
%%License: Ask me before reproducing/modifying, etc.


\documentclass{article}

%Make sure you have the file ShumanNote.scy in the same directory as
%this one. It has contains the style sheet for ECE111, and is needed
%to standardize the layout of LateX documents created for the class.
\usepackage{ShumanNotes} 
\usepackage{tikz}
\usepackage{program}
\usepackage{listings}
\pdfpagewidth 8.5in 
\pdfpageheight 11in

%This package is used to line up pictures 
\usepackage{graphicx}
\usepackage{fancyvrb}
\usepackage{listings}
%allows cursive font
%\usepackage{amsmath}

%allows hyperlinks 
%\usepackage{hyperref}

\newcommand{\HRule}{\rule{\linewidth}{0.5mm}} 

\lhead{Homework 5}

\begin{document}

%% These commands allow me to use cursive letter for things such as
%% length.  Note that on ubuntu linux, this required installation of
%% the package 'texlive-fonts-extra'. 
%% Taken from
%% http://www.latex-community.org/forum/viewtopic.php?f=5&t=1404&start=0
\newenvironment{frcseries}{\fontfamily{frc}\selectfont}{}
\newcommand{\textfrc}[1]{{\frcseries#1}}
\newcommand{\mathfrc}[1]{\text{\textfrc{#1}}}

\section{Overview}
The audio synthesizer is able to generate an audio output based on the user input.  The functionality of the circuit can be broken down into several subcomponents.  These include a microcontroller to interface with the external components, a power supply, EEPROM to dynamically store/load pre-defined user settings, a LCD output, a DAC, and an audio amplifier to provide more output power to the speaker.


\section{T-16 Audio Synthesizer Level 0 Diagram}
\centerimage{\Huge Diagram Goes Here}{Level 0 Diagram}{fullzero}

\begin{tabular}{|p{1in}|p{5in}|}
\hline
\emph{Module} & T-16 Audio Synthesizer \\
\hline
\emph{Inputs}& Lorem Ipsum\\
\hline
\emph{Outputs}& Dolor Sic \\ 
\hline
\emph{Functionality}& Bacon\\
\hline
\end{tabular}

\section{T-16 Audio Synthesizer Level 1 Diagram}
\centerimage{\Huge Diagram Goes Here}{Level 1 Diagram}{fullone}


\section{LCD Level 0 Diagram}
\centerimage{\Huge Diagram Goes Here}{Level 0 Diagram}{LCD}

\begin{tabular}{|p{1in}|p{5in}|}
\hline
\emph{Module} & LCD \\
\hline
\emph{Inputs}& Lorem Ipsum\\
\hline
\emph{Outputs}& Dolor Sic \\ 
\hline
\emph{Functionality}& Bacon\\
\hline
\end{tabular}

\section{Power Supply Level 0 Diagram}
\centerimage{\Huge Diagram Goes Here}{Level 0 Diagram}{PSU}

\begin{tabular}{|p{1in}|p{5in}|}
\hline
\emph{Module} & Power Supply \\
\hline
\emph{Inputs}& Lorem Ipsum\\
\hline
\emph{Outputs}& Dolor Sic \\ 
\hline
\emph{Functionality}& Bacon\\
\hline
\end{tabular}

\section{EEPROM Level 0 Diagram}
\centerimage{\Huge Diagram Goes Here}{Level 0 Diagram}{eeprom}

\begin{tabular}{|p{1in}|p{5in}|}
\hline
\emph{Module} & EEPROM \\
\hline
\emph{Inputs}& Lorem Ipsum\\
\hline
\emph{Outputs}& Dolor Sic \\ 
\hline
\emph{Functionality}& Bacon\\
\hline
\end{tabular}

\section{Microcontroller Level 0 Diagram}
\centerimage{\Huge Diagram Goes Here}{Level 0 Diagram}{micro}

\begin{tabular}{|p{1in}|p{5in}|}
\hline
\emph{Module} & Microcontroller \\
\hline
\emph{Inputs}& Lorem Ipsum\\
\hline
\emph{Outputs}& Dolor Sic \\ 
\hline
\emph{Functionality}& Bacon\\
\hline
\end{tabular}

\section{DAC Level 0 Diagram}
\centerimage{\Huge Diagram Goes Here}{Level 0 Diagram}{dac}

\begin{tabular}{|p{1in}|p{5in}|}
\hline
\emph{Module} & DAC \\
\hline
\emph{Inputs}& Lorem Ipsum\\
\hline
\emph{Outputs}& Dolor Sic \\ 
\hline
\emph{Functionality}& Bacon\\
\hline
\end{tabular}

\section{Audio Amplifier Level 0 Diagram}
\centerimage{\Huge Diagram Goes Here}{Level 0 Diagram}{audio}

\begin{tabular}{|p{1in}|p{5in}|}
\hline
\emph{Module} & Audio Amplifier \\
\hline
\emph{Inputs}& Lorem Ipsum\\
\hline
\emph{Outputs}& Dolor Sic \\ 
\hline
\emph{Functionality}& Bacon\\
\hline
\end{tabular}

\section{EEPROM UML Use Cases}

\subsection{Save Synth State}
\begin{tabular}{|p{1in}|p{5in}|}
\hline
\textbf{Use Case} & Save\\
\hline
\textbf{Description} & Saves the current synthesizer state into the synthesizer core memory.\\
\hline
\textbf{Actors} & I$^2$C, EEPROM, synthesizer core\\
\hline
\textbf{Assumptions} & There will be the correct amount of power supplied to the EEPROM and microcontroller.  In addition, there must be usable storage space to save in EEPROM.\\
\hline
\textbf{Issues} & EEPROM is not yet fully implemented and the exact location to the the memory is not yet defined.\\
\hline
\textbf{Steps} & \begin{enumerate}
\item Read the current state of the synthesizer.
\item Iterate over every byte of state data and send to the EEPROM via I$^2$C.
\end{enumerate}\\
\hline
\end{tabular}

\subsection{Load Synth State}
\begin{tabular}{|p{1in}|p{5in}|}
\hline
\textbf{Use Case} & Load\\
\hline
\textbf{Description} & Loads pre-saved synthesizer state into synthesizer core memory.\\
\hline
\textbf{Actors} & I$^{2}$C, EEPROM, synthesizer core\\
\hline
\textbf{Assumptions} & There will be the correct amount of power supplied to the EEPROM and microcontroller.  In addition, there will be a valid state to load.\\
\hline
\textbf{Issues} & asdf\\
\hline
\textbf{Steps} & \begin{enumerate}
\item
\end{enumerate}\\
\hline
\end{tabular}

\subsection{Load Splash Screen Image}
\begin{tabular}{|p{1in}|p{5in}|}
\hline
\textbf{Use Case} & Splash Screen Image\\
\hline
\textbf{Description} & Loads the Womprats Logo splash screen into memory.\\
\hline
\textbf{Actors} & LCD driver, EEPROM, I$^2$C\\
\hline
\textbf{Assumptions} & There will be the correct amount of power supplied to the EEPROM and microcontroller.  In addition, the image bits to be loaded will always remain intact.\\
\hline
\textbf{Issues} & asdf\\
\hline
\textbf{Steps} & \begin{enumerate}
\item
\end{enumerate}\\
\hline
\end{tabular}

\end{document}
