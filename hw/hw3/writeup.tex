%%MaD.tex - Notes taken for Materials and Devices Lecture
%%Author: Andy Goetz
%%Date Modified: 10-7-09
%%License: Ask me before reproducing/modifying, etc.


\documentclass{article}

%Make sure you have the file ShumanNote.scy in the same directory as
%this one. It has contains the style sheet for ECE111, and is needed
%to standardize the layout of LateX documents created for the class.
\usepackage{ShumanNotes} 
\usepackage{tikz}
\usepackage{program}
\usepackage{listings}
\pdfpagewidth 8.5in 
\pdfpageheight 11in

%This package is used to line up pictures 
\usepackage{graphicx}
\usepackage{fancyvrb}
\usepackage{listings}
%allows cursive font
%\usepackage{amsmath}

%allows hyperlinks 
%\usepackage{hyperref}

\newcommand{\HRule}{\rule{\linewidth}{0.5mm}} 

\lhead{Product Design Specification}

\begin{document}

%% These commands allow me to use cursive letter for things such as
%% length.  Note that on ubuntu linux, this required installation of
%% the package 'texlive-fonts-extra'. 
%% Taken from
%% http://www.latex-community.org/forum/viewtopic.php?f=5&t=1404&start=0
\newenvironment{frcseries}{\fontfamily{frc}\selectfont}{}
\newcommand{\textfrc}[1]{{\frcseries#1}}
\newcommand{\mathfrc}[1]{\text{\textfrc{#1}}}

Below is the specification of the Audio Synthesizer:

\section{Power Supply}

\begin{tabular}{|p{3in}|p{3in}|}
\hline
Engineering Requirement & Justification \\
\hline
\end{tabular}

\section{Sound Synthesis}

\begin{tabular}{|p{3in}|p{3in}|}
\hline
Engineering Requirement & Justification \\
\hline
\end{tabular}


\section{User Interface}

\begin{tabular}{|p{3in}|p{3in}|}
\hline
Engineering Requirement & Justification \\
\hline
\end{tabular}

\section{Case}

\begin{tabular}{|p{3in}|p{3in}|}
\hline
Engineering Requirement & Justification \\
\hline
\end{tabular}


\section{Sound Hardware}

\begin{tabular}{|p{3in}|p{3in}|}
\hline
Engineering Requirement & Justification \\
\hline
\end{tabular}


\section{Channel Interface}

The inputs of the audio synthesizer are known as ``Channels'' They are
a mix of analog and digital inputs. The following is a list of
specifications for these channels:

\vspace {.3in}

\begin{tabular}{|p{3in}|p{3in}|}
\hline
Engineering Requirement & Justification \\
\hline

The audio synthesizer must have channels that are permanently
connected to the design, as well as modular, external sensors. & In
order to remain flexible, the system must support external inputs,
however, the device has to be useable by itself, meaning that it must
have channels that are built into the unit. \\

\hline

The External Interface must be capable of supplying 25 mA per channel.
& In order to support a multitude of inputs, the external interfaces
of the Audio Synth need to be able to power a large range of sensors.\\

\hline

The channel external interface must provide a 3.3 volt source & In
order to support a wide range of sensors, the audio synth must provide
a power source in its external inputs. \\

\hline

Each external channel of the audio interface must provide a digital
signal, and an analog signal. & In order to be useful, a digital
signal is needed to gate synthesizer output, while an analog input is
needed to modulate the amplitude or frequency of the synthesizer.\\

\hline 

The external interface of the audio synth must filter out all
frequencies above 20Hz. & It is important that high frequency noise on
the inputs not cause ``humming'' on the output of the digital
synthesizer.\\

\hline

\end{tabular}


\end{document}
