%%MaD.tex - Notes taken for Materials and Devices Lecture
%%Author: Andy Goetz
%%Date Modified: 10-7-09
%%License: Ask me before reproducing/modifying, etc.


\documentclass{article}

%Make sure you have the file ShumanNote.scy in the same directory as
%this one. It has contains the style sheet for ECE111, and is needed
%to standardize the layout of LateX documents created for the class.
\usepackage{ShumanNotes} 
\usepackage{tikz}
\usepackage{program}
\usepackage{listings}
\pdfpagewidth 8.5in 
\pdfpageheight 11in

%This package is used to line up pictures 
\usepackage{graphicx}
\usepackage{fancyvrb}
\usepackage{listings}
%allows cursive font
%\usepackage{amsmath}

%allows hyperlinks 
%\usepackage{hyperref}

\newcommand{\HRule}{\rule{\linewidth}{0.5mm}} 

\lhead{Product Design Specification}

\begin{document}

%% These commands allow me to use cursive letter for things such as
%% length.  Note that on ubuntu linux, this required installation of
%% the package 'texlive-fonts-extra'. 
%% Taken from
%% http://www.latex-community.org/forum/viewtopic.php?f=5&t=1404&start=0
\newenvironment{frcseries}{\fontfamily{frc}\selectfont}{}
\newcommand{\textfrc}[1]{{\frcseries#1}}
\newcommand{\mathfrc}[1]{\text{\textfrc{#1}}}

Below is the specification of the Audio Synthesizer:

\section{Power Supply}

\begin{tabular}{|p{3in}|p{3in}|}
\hline
Engineering Requirement & Justification \\
\hline
The power supply must be capable of delivering a minimum of 400 mA current at a constant voltage level of 3v3.&In order to support the overall power requirements of the hardware, the power supply must supply the anticipated amount that the entire circuit will consume.\\
\hline
 Fuses must be present in the power supply.&For safety reasons, there should be circuit protection involved when transferring mains power to the device.  The fuse provides limited built-in internal protection from any harmful surges in power.\\
\hline
The power supply must be passively-cooled.&There must be as little noise interference produced from the device as possible.  Passive cooling methods are quieter than active cooling methods.\\
\hline
The power supply must be powered from the wall.&The battery/wall-wart combination involves extra circuit complication and the missing capability of a battery does not impact the primary objective of the device.  Thus, powering from the wall is sufficient for the intended application.\\
\hline
\end{tabular}

\section{Sound Synthesis}

\begin{tabular}{|p{3in}|p{3in}|}
\hline
Engineering Requirement & Justification \\
\hline
The sound synthesis must produce a sample rate of 44.1 kHz&
Since the human ear can hear frequencies up to ~20 kHz, the Nyquist rate 
determines that to be able to create waveforms in these higher frequencies, 
we need a sampling rate at least twice our highest output frequency. \\
\hline
The sound synthesis must produce 3 waveform variants: Square, Triange, and Sawtooth&
These 3 waveforms allow for many individual musical tones and timbres to be generated. Combinations of these 3
waveforms allow for many complex sounds to be generated.\\ 
\hline
The sound synthesis must support a gating feature for on/off control of oscillators&
Because some musical sounds lend themselves to shorter percussive sounds, gating allows a momentary
generation of the waveform to handle these types of instruments\\
\hline
The sound synthesis must allow frequency and amplitude parameter control of the oscillators from the 
analog inputs, as well as from the UI&
Allowing parameters to be fixed values from the UI, or driven in real-time from external inputs provides the
opportunity for complex combinations of waveforms from limited physical controls. \\
\hline
The sound synthesis must allow runtime software-defined routing of inputs to oscillator parameters&
Because there are more parameters to control than there are physical inputs to the device, there needs
to be flexibility in allowing the user to map physical inputs to the internal synthesis modules he or she
wishes to control.\\
\hline
The sound synthesis must be capable of creating output frequencies from 0-22 kHz&
The human ear can hear frequencies from approximately 20 Hz to 20 kHz, and musical notes throughout history
have spanned this range, it follows that our device should also cover all of these frequencies.\\
\hline
\end{tabular}

\section{User Interface}

\begin{tabular}{|p{3in}|p{3in}|}
\hline
Engineering Requirement & Justification \\
\hline
\end{tabular}

\section{Case}

\begin{tabular}{|p{3in}|p{3in}|}
\hline
Engineering Requirement & Justification \\
\hline
\end{tabular}


\section{Sound Hardware}

\begin{tabular}{|p{3in}|p{3in}|}
\hline
Engineering Requirement & Justification \\
\hline
The audio output of the device must support mono speaker and line out.&Mono speaker will provide enough sound output for the client to use the device.  The line out feature provides an additional option for headphones which may be desired when playing the device for personal use.\\
\hline
The audio output of the device must output a minimum of 1/2 W power.&It is important that the device be able to output audio with enough sound volume.  An output of 1/2 W is enough to drive a mono 8${\Omega}$ resistance speaker.\\
\hline
The audio output of the device must be passively cooled.&Noise interference with the main sound output of the device is undesired.  Thus, using passive cooling will minimize any noise from interfering with the desired output sound of the device.\\
\hline
\end{tabular}


\section{Channel Interface}

The inputs of the audio synthesizer are known as ``Channels'' They are
a mix of analog and digital inputs. The following is a list of
specifications for these channels:

\vspace {.3in}

\begin{tabular}{|p{3in}|p{3in}|}
\hline
Engineering Requirement & Justification \\
\hline

The audio synthesizer must have channels that are permanently
connected to the design, as well as modular, external sensors. & In
order to remain flexible, the system must support external inputs,
however, the device has to be useable by itself, meaning that it must
have channels that are built into the unit. \\

\hline

The External Interface must be capable of supplying 25 mA per channel.
& In order to support a multitude of inputs, the external interfaces
of the Audio Synth need to be able to power a large range of sensors.\\

\hline

The channel external interface must provide a 3.3 volt source & In
order to support a wide range of sensors, the audio synth must provide
a power source in its external inputs. \\

\hline

Each external channel of the audio interface must provide a digital
signal, and an analog signal. & In order to be useful, a digital
signal is needed to gate synthesizer output, while an analog input is
needed to modulate the amplitude or frequency of the synthesizer.\\

\hline 

The external interface of the audio synth must filter out all
frequencies above 20Hz. & It is important that high frequency noise on
the inputs not cause ``humming'' on the output of the digital
synthesizer.\\

\hline

\end{tabular}


\end{document}
