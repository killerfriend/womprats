%%MaD.tex - Notes taken for Materials and Devices Lecture
%%Author: Andy Goetz
%%Date Modified: 10-7-09
%%License: Ask me before reproducing/modifying, etc.


\documentclass{article}

%Make sure you have the file ShumanNote.scy in the same directory as
%this one. It has contains the style sheet for ECE111, and is needed
%to standardize the layout of LateX documents created for the class.
\usepackage{ShumanNotes} 
\usepackage{tikz}
\usepackage{program}
\usepackage{listings}
\pdfpagewidth 8.5in 
\pdfpageheight 11in

%This package is used to line up pictures 
\usepackage{graphicx}
\usepackage{fancyvrb}
\usepackage{listings}
%allows cursive font
%\usepackage{amsmath}

%allows hyperlinks 
%\usepackage{hyperref}

\newcommand{\HRule}{\rule{\linewidth}{0.5mm}} 

\lhead{Homework 2}

\begin{document}

%% These commands allow me to use cursive letter for things such as
%% length.  Note that on ubuntu linux, this required installation of
%% the package 'texlive-fonts-extra'. 
%% Taken from
%% http://www.latex-community.org/forum/viewtopic.php?f=5&t=1404&start=0
\newenvironment{frcseries}{\fontfamily{frc}\selectfont}{}
\newcommand{\textfrc}[1]{{\frcseries#1}}
\newcommand{\mathfrc}[1]{\text{\textfrc{#1}}}


\section{Group Overview}
Our group was formed with Andy Goetz, Bradon Kanyid, Jackson Pugh, and
Kevin Riedl. Our group has a wiki located at
\url{https://projects.cecs.pdx.edu/projects/fa2012ece411-womprats/wiki}.
\section{Project Overview}

We evaluated 4 separate ideas for our practicum: An audio synthesizer,
a flood sensor, a space wars clone, and a gameboy/MIDI interface.

\subsection{Audio Synthesizer}
The audio synth project would be a simple synthesizer that does
software-based processing and synthesis of waveforms from various
input sources, such as possibly potentiometers, photodiodes, and other
tactile sensors. There may be a user interface for modifying effects,
chaining together inputs, and calibration of inputs. The output would
be a built-in speaker with an optional headphone/line-out. This would
likely have two power options - portable and fixed. MIDI input is a
possibility as well. So basically the flow would be "sensor inputs ->
modifications -> synthesis -> audio outputs", or something along those
lines.
\subsection{Flood Sensor}
The Flood sensor project is a rather simple device that would alert
the occupants of a house/building when a flood would happen. This is
similar in idea to a smoke alarm or a carbon monoxide detector. There
are several different ways this could be implemented and could have
varying degrees of complication depending on how we want to implement
the alert system.
\subsection{Space Wars Clone}
In this project, we would construct a clone of the classic videogame,
Space Wars. This would involve creating hardware to interact with
joysticks, as well as hardware and software to display the game on an
NTSC television. 

\subsection{Gameboy/MIDI interface}

The Gameboy MIDI project allows the user to hook up a MIDI
controller/device to a Gameboy (via the link port) and play 8-bit
notes through the speakers. Specifically, when the gameboy is turned
on, the device can take in MIDI control signals, which it processes
and then outputs as a tune. Thus, you can use the Gameboy as musical
instrument to play classic 8-bit sounds.

\newpage

\section{Decision Matrix}
We created a decision matrix to evaluate these ideas. We did not use
individual weights for the criteria in our decision matrix: all of the
scores ranged from 1 to 5, with 5 being more desirable.

\centerimage{
\begin{tabular}{l|p{.7in}p{0.8in}p{0.8in}p{01in}}
& Audio Synthesizer & Flood Sensor & Space Wars & Gameboy MIDI Interface \\
\hline
Mechanical Complexity &4 &2 &5 &3 \\
Cheap to Make& 3& 5& 4&4 \\
Open Source / Free & 5& 5& 5& 3\\
Gracefully Degradeable & 5& 1& 3& 1\\
Visible Circuit & 4& 5& 3& 2\\
Plug 'n Play & 5& 5& 5& 4\\
Presentable & 5& 4& 3&4 
\end{tabular}
}{Decision Matrix}{decmatrix}

\section{Decision Matrix Criteria}
The dimension of
our matrix were Mechanical Complexity, Cheap to Make, Open Source,
Gracefully Degradeable, Visible Circuit, Plug n' Play, and
Presentable. Scores ranged from 1 to 5, with 5 being better. More
detail about these dimensions can be found below.
\subsection {Mechanical Complexity}
This dimension measures how complex the device would be to manufacture
mechanically. For example, if a project only requires a bare circuit
board, it would score a 5 in this category. If it required a
custom-designed enclosure, it would score closer to a one. 
\subsection{Cheap to Make}
This dimension measures how expensive the device would be to
manufacture. Projects that only require cheap microcontrollers and
components would score a 5, while projects that require exotic
materials, expensive components, or existing objects (i.e. a game boy)
would score lower. 
\subsection{Open Source}
Does the project depend only on open source or well documented tools?
Open source tools tend to be better quality and easier to use as well.
Using an AVR microcontroller would score higher than a Xilinx
FPGA. Both of these are significantly easier than programming a game
boy.
\subsection{Gracefully Degradeable}
Can the scope of this project be easily scaled back if certain aspects are
unachievable? 
\subsection{Visible Circuit}
How easy is it to prototype this project before commiting to a PCB?
Are breakouts commonly available? Does the project depend on non-DIP
parts, or operate at a frequency too high for a breadboard?
\subsection{Plug n' Play}
Does this project require custom software to run on a host computer,
or will it work as soon as it is plugged into any other devices?
\subsection{Presentable}
How easy is this device to demonstrate without depending on external
resources? Can it be shown off in an interview situation?

\section{Details about individual projects}
Each project has its value on the decision matrix explained below. 

\subsection{Audio Synthesizer}
\subsubsection {Mechanical Complexity}
Since this project is fully independent of any other devices, it only consists
of our PCB, a few panel-mounted components, and the enclosure. The only 
mechanical complexity will be the input device interfacing, because these will
be plugged and unplugged often.
\subsubsection{Cheap to Make}
This project would be mostly composed of cheap digital ICs. However, there will
likely be an interface necessary, and that will mean some sort of LCD or similar
device.  
\subsubsection{Open Source}
We will be able to program this device independent of anything external to the
project, so any of the main processing units we may use have open source tools.
There doesn't seem to be any need for non-opensource tools for this concept.
\subsubsection{Gracefully Degradeable}
This project ideally would be capable of having multiple input sources to
manipulate the sound, and have many effects. However, if there are any unforseen
huge roadblocks, it should be easy to pare down the feature set.
\subsubsection{Visible Circuit}
In general, this project seems like it should be very easy to prototype. However,
since some of the specifics of this project are not specifically nailed down, we
lowered the score slightly. The other projects had a more understood level of
processing and peripheral requirements.
\subsubsection{Plug n' Play}
This project's device would be standalone and require nothing more than a charged
battery or wall power. This makes it score very highly in this category.
\subsubsection{Presentable}
We felt that this device would be very presentable to nearly anyone, because
everyone likes playing with things that make funny sounds, and it has no external
requirements to operate. It is fully independent.

\subsection{Flood Sensor}
\subsubsection {Mechanical Complexity}
This project would require a enclosure that would be water-proof due to the
fact that it is intended to function in a situation in which water would come
in contact with the device. Therefore it has a high mechanical complexity.
\subsubsection{Cheap to Make}
This project would require very few parts and the enclosure would be made to be
inexpensive. This would give us a very low cost.
\subsubsection{Open Source}
Since everything on the device would ideally be programmed by us and this type
of device could easily use a variety of different components, we would lean
toward using open source and free hardware tools.
\subsubsection{Gracefully Degradeable}
This is kind of a single purpose device, if it does not detect a flood, then
it is a failure. Therefore it is not gracefully degradable.
\subsubsection{Visible Circuit}
Nothing in this circuit would require high frequency signals or anything that 
could not be tested on a breadboard, thus it is easy to prototype.
\subsubsection{Plug n' Play}
Everything used on this project would be very easy to simply plug in and work.
\subsubsection{Presentable}
The design would require this device to be portable and easy to use, the only
exception being that to demonstrate it you would need some liquid to submerge 
some part of the device to simulate flood conditions.


\subsection{Space Wars}
\subsubsection {Mechanical Complexity}
This project did not have a very high mechanical complexity, as in a
pinch, it could be left as a bare board. 
\subsubsection{Cheap to Make}
This project is relatively cheap to make, as it would not require any
expensive components. The most expensive components would be the
controllers, and these could be had for less than 5 dollars. 
\subsubsection{Open Source}
This project could be made with completely open source
tools. Additionally, the source code for the Space Wars game is
available online.
\subsubsection{Gracefully Degradeable}
The project is fairly degradeable. If the Space Wars project was too
ambitious, we could always program a simpler game, such as
pong. However, if the NTSC output of controller input does not work,
the project would be dead in the water.
\subsubsection{Visible Circuit}
This project would be fairly easy to prototype. The Analog circuitry
for the NTSC output might be hard to do on a breadboard. 
\subsubsection{Plug n' Play}
This project would be very plug and play, just plug in the controllers and a TV.
\subsubsection{Presentable}
This project would be fairly presentable. However, it might be hard to
find a television in a conference room to plug in to. 

\subsection{Gameboy/MIDI Interface}
\subsubsection {Mechanical Complexity}
The mechanical complexity of a MIDI gameboy is relatively in-between
because it requires "special" tools to put together. It also requires
buying the gameboy which is special hardware.
\subsubsection{Cheap to Make}
The overall cost of a MIDI gameboy is cheap. The gameboy only costs
around \$20 and the other components combined would total to be less
than \$20. A total budget of \$50 or less is anticipated.
\subsubsection{Open Source}
While this project has been done before, the gameboy is a proprietary
device licensed under Nintendo as being closed source. Thus, there is
limited support only from other hobbyists and fellow hackers and a lot
of the work will require reverse engineering.
\subsubsection{Gracefully Degradeable}
This project is a 'do-or-die'. That is, everything must be implemented
in order for the project to be a success. Half of the MIDI can't be
non-functioning and the project still be considered a success. Thus,
no "features" can be removed without the project failing its primary
goal.
\subsubsection{Visible Circuit}
The gameboy's internal circuit is hard to get to because it is
enclosed in a tight container. The MIDI controller is also the same
way.
\subsubsection{Plug n' Play}
Everything is ready to work out of the box. There is no need to write
special firmware or install drivers. However, a MIDI controller is
required to control the input to the gamebody.
\subsubsection{Presentable}
The MIDI gameboy is relatively portable and does not require a bunch
of cables or accessories. This helps in bringing the device to an
interview and demonstrating its features. Overall, the hardest part
about this is the MIDI device used to hook up to the gameboy. The
gameboy itself is small and extremely portable.
\section {Project Overview}

In the end, we decided to develop the audio synth idea for our
practicum. We then created a high-level requirements document for this
project:

\subsection{Must}
\begin{itemize}
\item Modular DC Wall Power
\item Make sounds
\item Have output volume control
\item Have sound output (headphones/lineout)
\item Have LCD for UI
\end{itemize}
\subsection{Should}
\begin{itemize}
\item Modular Inputs 
\item Battery Power
\item Easy to maintain (easily openable)
\item Be handheld
\item Have speaker
\end{itemize}
\subsection{May}
\begin{itemize}
\item Inputs toggle from physical buttons
\item LEDs/LED graphs for frequency and intensity
\item Be useable in the dark
\item Memory to store instruments
\end{itemize}


\end{document}
